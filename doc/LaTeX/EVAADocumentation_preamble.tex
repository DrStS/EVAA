%%%%%%%%%%%%%%%%%%%%%%%%%%% S T A R T I N G %%%%%%%%%%%%%%%%%%%%%%%%%%%
%%%%%%%%%%%%%%%%%%%%%%%%%%%%%%%%%%%%%%%%%%%%%%%%%%%%%%%%%%%%%%%%%%%%%%%
%%%%%%%%%%%%%%%%%%%%%%%%%%  P R E A M B L E  %%%%%%%%%%%%%%%%%%%%%%%%%%
%%%%%%%%%%%%%%%%%%%%%%%%%%%%%%%%%%%%%%%%%%%%%%%%%%%%%%%%%%%%%%%%%%%%%%%
% Usefull documents
% http://tug.org/pracjourn/2008-1/mori/mori.pdf
% The overall rule is that symbols representing physical quantities (or variables) are italic, but symbols representing units, or labels, are roman.
%• Symbols for vectors and matrices are bold italic, symbols for tensors are sansserif bold italic.
%• The above rules apply equally to letter symbols from the Greek and the Latin alphabet.
% \mathbfit   -->  boldface italic       --> vector and matrix symbols
% \mathsfbfit --> sans-serif bold italic --> tensor symbols
\documentclass [10pt, twoside, openright, a4paper, final %showtrims
]{memoir}
\overfullrule=20mm
%\trimFrame
%\isopage[12]
%\isopage
% Din-A5 148mmx210mm
%\setlrmarginsandblock{20mm}{27mm}{*}
%\setulmarginsandblock{25mm}{19mm}{*}
\setlrmarginsandblock{24mm}{33mm}{*}
\setulmarginsandblock{30mm}{23mm}{*}
\checkandfixthelayout
\fixpdflayout
\raggedbottom
%\\%%%%%%%%%%%%%%%%%% LOAD PACKAGES %%%%%%%%%%%%%%%%%%%%%
%Get Umlaut support
\usepackage[utf8]{inputenc}
\usepackage[T1]{fontenc}
\usepackage{float}
%Get Support for US english
\usepackage[ngerman,english]{babel}
% Rotate pages part wise
\usepackage{pdflscape}
%Setting a date
\usepackage{datetime}
%Support for Graphics 
\usepackage{graphicx}
	\graphicspath{{gnuplot/basic/}{input/fig/}{unittests/fig/}{formulation/fig/}}%Set global search path
\usepackage{transparent}
\usepackage[utopia,cal=cmcal]{mathdesign} % --> Adobe Utopia for the whole document http://www.latex-community.org/forum/viewtopic.php?f=48&t=6989
%http://tex.stackexchange.com/questions/30550/force-upright-greek-letters-with-isomath
\usepackage[OMLmathrm,OMLmathbf]{isomath} % options define which alphabets will be loaded, i.e. if bold face font is not necessary, `OMLmathbf` can be ommitted.
\DeclareMathAlphabet\mathbfcal{OMS}{cmsy}{b}{n}
% activate={true,nocompatibility} - activate protrusion and expansion
% final - enable microtype; use "draft" to disable
% tracking=true, kerning=true, spacing=true - activate these techniques
% factor=1100 - add 10% to the protrusion amount (default is 1000)
% stretch=10, shrink=10 - reduce stretchability/shrinkability (default is 20/20)
\usepackage[activate={true,nocompatibility},final,tracking=true,kerning=true,spacing=true,factor=1000,stretch=20,shrink=20]{microtype}
\DeclareMicrotypeSet*[tracking]{my}% 
  { font = */*/*/sc/* }% 
\SetTracking{ encoding = *, shape = sc }{ 45 }
%erweiterter Mathemodus
\usepackage[fleqn]{amsmath}
\allowdisplaybreaks[1]
\renewcommand{\arraystretch}{1.5}
\setlength{\mathindent}{1.2\parindent}
% fleqn envirnoment
%\usepackage{nccmath}
% Porper differentials
\usepackage{commath}
% Copyright symbols
\usepackage{textcomp}
%Customizing lists
\usepackage{enumitem}
\usepackage{multirow}
\usepackage{wasysym} % defines  \Box
%cancel equations
\usepackage{cancel}
%Units
\usepackage{units}
\let\endleftbar\relax
\let\leftbar\relax
%Theorems
\usepackage[most]{tcolorbox}
\tcbset{listing engine=listings}
%Pseudo code
\usepackage[ruled,vlined,linesnumberedhidden,scleft,algochapter]{algorithm2e}
%\SetAlFnt{\small}
\newcommand\mycommfont[1]{\footnotesize\ttfamily\textcolor{gray50}{#1}}
\SetCommentSty{mycommfont}
\SetAlCapFnt{\small}
\SetAlCapNameFnt{\small}
%Fancyref
\let\figref\fref % use \figref for memoir's \fref
\let\fref\relax  % "undefine" \fref
%referencing
\usepackage[german,plain]{fancyref} 
%% START Eigene fancyrefs
%Theorem
\newcommand*{\fancyrefthmeoremlabelprefix}{thm}
%\frefformat{vario}{\fancyrefthmeoremlabelprefix}{\frefthmeoremname\fancyrefdefaultspacing#1#3}
%\Frefformat{vario}{\fancyrefthmeoremlabelprefix}{\Frefthmeoremname\fancyrefdefaultspacing#1#3}
%% END Eigene fancyrefs
%Inputdecks
\usepackage{listings}
%use coloring
\usepackage{xcolor}
%Changing Layout
%nomenclature
\usepackage[]{nomencl} 
\makenomenclature 
\renewcommand{\nomname}{List of Symbols and Abbreviations}
% rotate
\usepackage{rotating}
%% BibLaTeX
\usepackage{etoolbox}
\usepackage{enumitem}
\usepackage[autostyle,german=guillemets]{csquotes}
\renewcommand{\mkcitation}[1]{\footnote{#1}}
\renewcommand{\mktextquote}[6]{#1#2#4#3#5#6}
\usepackage[natbib=false,citestyle=numeric-comp,bibstyle=numeric-comp,sortcites=true,firstinits=true,backend=biber,maxbibnames=10,maxcitenames=1,url=false,defernumbers=true 
%,block=nbpar
]{biblatex}
\let\cite\textcite
\AtBeginBibliography{\raggedright}
\AtEveryBibitem{\clearfield{issn}}
\AtEveryCitekey{\clearfield{issn}}
%Break links in biblio
\appto{\bibsetup}{\emergencystretch=5em}
\bibliography{EVAADocumentation}
%Booktabs
\usepackage{booktabs}
%Longtable
\usepackage{longtable}
\newcounter{magicrownumbers}
\newcommand\rownumber{\stepcounter{magicrownumbers}\arabic{magicrownumbers}}
\usepackage{colortbl}
%//%%%%%%%%%%%%%%%%%% LOAD PACKAGES %%%%%%%%%%%%%%%%%%%%%
%\\%%%%%%%%%%%%%%%%%% SETTINGS %%%%%%%%%%%%%%%%%%%%%
%% START MICROTYPE
%% Keine "Schusterjungen"
\clubpenalty = 10000
%% Keine "Hurenkinder"
\widowpenalty = 10000
\displaywidowpenalty = 10000
%% END MICROTYPE
%% START NUMPRINT
% Print numbers \numprint{123456.134e123}’
\usepackage[]{numprint}
\npthousandsep{\,}
\npdecimalsign{.}
\npproductsign{\cdot}
\npunitseparator{\,}
%% END NUMPRINT
%% START TABULAR
%% Tabellen Anpassung nach Axel Reichert
\makeatletter
\newcolumntype{H}{>{\scriptsize\raggedright\arraybackslash}l}
\newcolumntype{G}[1]{%
>{\scriptsize\raggedright\hspace{0pt}}p{#1}%
}
\newcolumntype{C}{>{\scriptsize\raggedright\arraybackslash}l}
\newcolumntype{V}[1]{%
>{\scriptsize\raggedright\hspace{0pt}}p{#1}%
}
\makeatother
%% END TABULAR
%% START TITLE
\newcommand*{\DocumentTitle}[1]{\def\DocumentTitle{#1}}
\DocumentTitle{EVAA: Efficient Vehicle dynAmics simulAtor}
%% END TITLE
%\setlength{\parindent}{0mm}%no paragraph indentation
%% START define colors
\definecolor{shadecolor}{rgb}{0.7,0.8,0.9}
\definecolor{TUMhellblau}{HTML}{0073CF}
\definecolor{TUMhellhellhellblau}{HTML}{98C6EA}
\definecolor{Weiss}{HTML}{FFFFFF}
\definecolor{gray05}{gray}{0.95}
\definecolor{gray20}{gray}{0.80}
\definecolor{gray50}{gray}{0.50}
\definecolor{gray80}{gray}{0.20}
\definecolor{gray90}{gray}{0.10}
%% END define colors

%% START Kopf- und Fußzeilenanpassung
% The following code have been found in Peter Wilsons Memoir manual in chapter 7 (page127)
\makepagestyle{YourPagestyleName} % Create a new pagestyle
% Following code to edit the pagestyle
\makepsmarks{YourPagestyleName}{%
\nouppercaseheads
% This is where we specify what \leftmark and \rightmark contains
\createmark{chapter}{both}{shownumber}{}{\space}
\createmark{section}{right}{shownumber}{}{\space}
% Change "shownumber" to "nonumber" if you don't want
%   the chapter/section number displayed at the header.
\createplainmark{toc}{both}{\contentsname}
\createplainmark{lof}{both}{\listfigurename}
\createplainmark{lot}{both}{\listtablename}
\createplainmark{bib}{both}{\bibname}
\createplainmark{index}{both}{\indexname}
\createplainmark{glossary}{both}{\glossaryname}}
% Might want to keep those, see the manual for further information.

% The following is where you tweek your header
\makeevenhead{YourPagestyleName}    {\normalfont\leftmark}{}{}
\makeoddhead{YourPagestyleName}{}{}{\normalfont\rightmark}
\makeevenfoot{YourPagestyleName}{%      % For verso pages
\normalfont\thepage}{}{}
\makeoddfoot{YourPagestyleName}{%       % For recto pages
}{}{\normalfont\thepage}

% Activate your new pagestyle
\pagestyle{YourPagestyleName}
%% END Kopf- und Fußzeilenanpassung

%% START Caption
\changecaptionwidth
\captionwidth{0.8\linewidth}
\setfloatadjustment{figure}{\centering}
\setfloatadjustment{table}{\footnotesize\centering}
\captionnamefont{\bfseries\footnotesize} %caption label
\captiontitlefont{\footnotesize}
\captionstyle{\centering}
\subcaptionstyle{\centering}
\newfixedcaption{\figcaption}{figure}
%% END Caption

%% START set depth of TOC
\renewcommand{\cftsectiondotsep}{\cftnodots}
\maxtocdepth{subsection}
\setsecnumdepth{subsubsection}
\maxsecnumdepth{subsubsection}
\makeatletter
\renewcommand{\@pnumwidth}{3em} 
\renewcommand{\@tocrmarg}{4em}
\makeatother
%% END set depth of TOC

%% START SOURCE CODE QUOTING SETTINGS
\renewcommand{\lstlistlistingname}{List of Source Code Snippets}
\renewcommand{\lstlistingname}{Source Code Snippet}
\lstset{ %
%classoffset=0,
basicstyle=\tiny\ttfamily,
keywordstyle=\color{TUMhellhellhellblau}\ttfamily,
stringstyle=\color{violet}\ttfamily,
commentstyle=\color{gray50},
identifierstyle=\color{black},
numbers=left,                   % where to put the line-numbers
numberstyle=\tiny,              % the size of the fonts that are used for the line-numbers
stepnumber=2,                   % the step between two line-numbers. If it's 1 each line 
                                % will be numbered
firstline=2,
numberfirstline=false,
firstnumber=2,                                
numbersep=5pt,                  % how far the line-numbers are from the code
backgroundcolor=\color{gray05}, % choose the background color. You must add \usepackage{color}
rulecolor=\color{gray80},
rulesepcolor=\color{gray80},
showspaces=false,               % show spaces adding particular underscores
showstringspaces=false,         % underline spaces within strings
showtabs=false,                 % show tabs within strings adding particular underscores
frame=none,                     % adds a frame around the code
framerule=0pt,                  % remove box
tabsize=2,	                    % sets default tabsize to 2 spaces
captionpos=t,                   % sets the caption-position to bottom
breaklines=true,                % sets automatic line breaking
extendedchars=true,
breakatwhitespace=false,        % sets if automatic breaks should only happen at whitespace
title=\lstname,                 % show the filename of files included with \lstinputlisting;
                                % also try caption instead of title
escapeinside={\%*}{*)},         % if you want to add a comment within your code
morekeywords={*,...}            % if you want to add more keywords to the set
}
\lstloadlanguages{% Check documentation for further languages ...
		 C
 }
%% END SOURCE CODE QUOTING SETTINGS 

%% START Tabellen Anpassung nach Axel Reichert
%  not needed anymore use numprint macros n and N
%% END Tabellen Anpassung nach Axel Reichert

%% START Anpassung der Kapitelüberschriften
\usepackage{calc,soul}
\makeatletter
\newlength\dlf@normtxtw
\setlength\dlf@normtxtw{\textwidth}
\def\myhelvetfont{\def\sfdefault{mdput}}
\newsavebox{\feline@chapter}
\newcommand\feline@chapter@marker[1][4cm]{%
\sbox\feline@chapter{%
\resizebox{!}{#1}{\fboxsep=1pt%
\colorbox{TUMhellblau}{\color{white}\bfseries\scshape\thechapter}%
}}%
\rotatebox{90}{%
\resizebox{%
\heightof{\usebox{\feline@chapter}}+\depthof{\usebox{\feline@chapter}}}%
{!}{\scshape\so\@chapapp}}\quad%
\raisebox{\depthof{\usebox{\feline@chapter}}}{\usebox{\feline@chapter}}%
}
\newcommand\feline@chm[1][4cm]{%
\sbox\feline@chapter{\feline@chapter@marker[#1]}%
\makebox[0pt][l]{% aka \rlap
\makebox[1cm][r]{\usebox\feline@chapter}%
}}
\makechapterstyle{daleif1}{
\renewcommand\chapnamefont{\normalfont\Large\scshape\raggedleft\so}
\renewcommand\chaptitlefont{\normalfont\huge\bfseries\scshape\color{TUMhellblau}}
\renewcommand\chapternamenum{}
\renewcommand\printchaptername{}
\renewcommand\printchapternum{\null\hfill\feline@chm[2.5cm]\par}
\renewcommand\afterchapternum{\par\vskip\midchapskip}
\renewcommand\printchaptertitle[1]{\chaptitlefont\raggedleft ##1\par}
\makeoddfoot{plain}{}{}{\thepage}
\makeevenfoot{plain}{}{}{\thepage}
}
\makeatother
\chapterstyle{daleif1}

\epigraphposition{flushleft}
% Change footer in combination with epigraphhead
\makeatletter
\renewcommand{\epigraphhead}[2][95]{%
   \def\@epitemp{\begin{minipage}{\epigraphwidth}#2\end{minipage}}
   \def\ps@epigraph{\ps@plain
     \let\@mkboth\@gobbletwo
     \@epipos
     \if@epirhs
       \def\@oddhead{\hfil\begin{picture}(0,0)
                          \put(0,-#1){\makebox(0,0)[r]{\@epitemp}}
                          \end{picture}}
     \else
       \if@epicenter
         \def\@oddhead{\hfil\begin{picture}(0,0)
                            \put(0,-#1){\makebox(0,0)[b]{\@epitemp}}
                            \end{picture}\hfil}
       \else
         \def\@oddhead{\begin{picture}(0,0)
                            \put(0,-#1){\makebox(0,0)[l]{\@epitemp}}
                            \end{picture}\hfil}
       \fi
     \fi
     \let\@evenhead\@oddhead}
   \thispagestyle{epigraph}}
\makeatother
%% END Anpassung der Kapitelüberschriften

%% START Anpassung der Theoreme
\newtcbtheorem[auto counter,number within=chapter,list inside=exam]{myExample}{Example}{
    %float=!htb,
	breakable,
	enhanced,
	arc=0pt,
	outer arc=0pt,
	coltitle=black,
	fonttitle=\footnotesize\bfseries,
	fontupper=\noindent\tiny,
	colback=gray05,
	colframe=gray20,
	left=7pt,
	lefttitle=7pt,
	boxsep=2pt,
	right=7pt,
}{eg}
\newtcbtheorem[auto counter,number within=chapter,list inside=theo]{myTheorem}{Theorem}{
	unbreakable,
	enhanced,
	arc=0pt,
	outer arc=0pt,
	coltitle=black,
	fonttitle=\small\bfseries,
	fontupper=\noindent,
	colback=gray05,
	colframe=gray20,
	left=7pt,
	lefttitle=7pt,
	boxsep=2pt,
	right=7pt,
}{thm}
\newtcbtheorem[auto counter,number within=chapter,list inside=defi]{myDefinition}{Definition}{
	breakable,
	enhanced,
	arc=0pt,
	outer arc=0pt,
	coltitle=black,
	fonttitle=\small\bfseries,
	fontupper=\noindent,
	colback=gray05,
	colframe=gray20,
	left=7pt,
	lefttitle=7pt,
	boxsep=2pt,
	right=7pt,
}{def}
\makeatletter
\newcommand\notefont{\normalfont}
\def\tcb@theo@title#1#2#3{%
  \ifdefempty{#2}{\setbox\z@=\hbox{#1}}{\setbox\z@=\tcb@theo@form{#1}{#2}}%
  \def\temp@a{#3}%
  \ifx\temp@a\@empty\relax%
    \unhbox\z@\kvtcb@terminatorsign%
  \else%
    \setbox\z@=\hbox{\unhbox\z@\kvtcb@separatorsign\ }%
    \hangindent\wd\z@%
    \hangafter=1%
    \mbox{\unhbox\z@}\kvtcb@desc@delim@left\notefont#3\kvtcb@desc@delim@right\kvtcb@terminatorsign%
  \fi%
}
\makeatother

\newtcbinputlisting[auto counter,number within=chapter]{\mylisting}[2][]{%
	listing file={#2},
	title=\textbf{\thetcbcounter~Listing:} {#2},
	enhanced,
	arc=0pt,
	outer arc=0pt,
	coltitle=black,
	fonttitle=\small,
	fontupper=\noindent,
	colback=gray05,
	colframe=gray20,
	left=0pt,
	lefttitle=7pt,
	boxsep=2pt,
	right=7pt,
	listing only,
	breakable,
	listing options={language=C},
	#1}
%% END Anpassung der Theoreme

%% START Eigene Math macros
\newcommand\leftidx[3]{%
{\vphantom{#2}}#1#2#3%
}
\makeatletter
\g@addto@macro\footnotesize{%
  \setlength\abovedisplayskip{2pt}
  \setlength\belowdisplayskip{2pt}
  \setlength\abovedisplayshortskip{2pt}
  \setlength\belowdisplayshortskip{2pt}
}
\makeatother
%% END Eigene Math macros

%% START HYPERREF
%Links
\usepackage[plainpages=false,pdfpagelabels,pdfview=Fit,pdftex,hypertexnames=false]{hyperref} 
\newsubfloat{figure}% Allow subfloats in figure environment
\hypersetup{   
	bookmarksopen=false,    % expand tree
	unicode=false,          % non-Latin characters in Acrobat’s bookmarks
	pdftoolbar=false,       % show Acrobat’s toolbar?
	pdfmenubar=true,        % show Acrobat’s menu?
	pdffitwindow=false,     % window fit to page when opened
	pdfstartview={FitH},    % fits the width of the page to the window
	pdftitle={\DocumentTitle},   % title
	pdfauthor={Stefan Sicklinger}, % author
	pdfsubject={Projektdokumentation},       % subject of the document
	pdfkeywords={Körperschall; Vibroakustik;},         % list of keywords
	pdfnewwindow=true,      % links in new window
	colorlinks=true,        % false: boxed links; true: colored links
	linkcolor=black,        % color of internal links
	citecolor=black,        % color of links to bibliography
	filecolor=black,        % color of file links
	urlcolor=black          % color of external links
}
%% END HYPERREF
%//%%%%%%%%%%%%%%%%%% SETTINGS %%%%%%%%%%%%%%%%%%%%%

%\\%%%%%%%%%%%%%%%%%% OWN COMMANDS %%%%%%%%%%%%%%%%%%%%%
\newcommand{\itab}[1]{\hspace{0em}\rlap{#1}}
\newcommand{\tab}[1]{\hspace{.35\textwidth}\rlap{#1}}
\newcommand{\Reyn}{\operatorname{\mathit{Re}}}
\newcommand{\Strn}{\operatorname{\mathit{Sr}}}
\DeclareMathOperator{\colspan}{colspan}
\DeclareMathOperator{\diag}{diag}
%//%%%%%%%%%%%%%%%%%% OWN COMMANDS %%%%%%%%%%%%%%%%%%%%%
\hyphenation{}
%//%%%%%%%%%%%%%%%%%% OWN COMMANDS %%%%%%%%%%%%%%%%%%%%%
\newcommand{\FLpar}[2] { % a simple partial derivation - F. L.
	\frac{\partial #1}{\partial #2}
}
\newcommand{\FLexpr}[3] {
	\FLpar{#1}{#2}\FLpar{#2}{#3}+\FLpar{#1}{#3}
}
%%%%%%%%%%%%%%%%%%%%%%%%%%%   E N D I N G   %%%%%%%%%%%%%%%%%%%%%%%%%%%
%%%%%%%%%%%%%%%%%%%%%%%%%%%%%%%%%%%%%%%%%%%%%%%%%%%%%%%%%%%%%%%%%%%%%%%
%%%%%%%%%%%%%%%%%%%%%%%%%%  P R E A M B L E  %%%%%%%%%%%%%%%%%%%%%%%%%%
%%%%%%%%%%%%%%%%%%%%%%%%%%%%%%%%%%%%%%%%%%%%%%%%%%%%%%%%%%%%%%%%%%%%%%%